\documentclass{article}
\usepackage{amsmath}

\begin{document}

\title{Hourglassin'}
\author{Rube}
\author{Seth}

\section{Introduction}
We be MIG'n and hourglassin'.

\section{Theory}
\subsection{MIG}\label{MIG}
Use least-squares to show
$\lim_{n\rightarrow\inf}err=0$.

\subsection{Hourglassin'}
\subsubsection{Motivation}
Because available sensor models (such as RPC) often lack a meaningful
error model to input to MIG, and for computational simplicity, we
introduce a heuristic approach as an alternative to rigorous
least-squares MIG. Although when visualized 'up close' to the true
answer, any particular tangle of rays may seem unclear about where the
ground point should be localized, from a global perspective, the ray
bundle will be something like a cone, widening to the cluster of
satellite (or airborne) perspective centers, narrowing at the ground
point, and widening again beyond the ground point. We seek the answer
at the narrowest point of the cone. In an ideal case, the ray bundle
will intersect at a single point, which has cross-sectional area of
0. In a real-world case, the bundle will appear as a cone with a `fat'
intersection. Thus the name `hourglassing' to motivate the heuristic
technique.

Given a bundle of rays, we can intersect the bundle with planes of
various heights, compute the collection of intersections of the ray
bundle with each height plane, measure the spread of the 2-D
distribution of points, and choose the plane with the least spread to
be the solution for the height of the desired ground point. For the
horizontal location of the ground point, the natural choice is the
mean of the intersection points in the chosen plane.

Rather than attacking this problem with a brute force search for the
spread-minimizing by computing intersection sets at very many heights,
and slicing height space sufficiently thin to achieve a desired
vertical resolution, we attempt some theoretical underpinnings for
this technique that will allow a more efficient and precise solution,
and help motivate a meaningful estimate of the error of the resulting
ground point.

\subsubsection{Computing height of minimum spread}
Assume two heights $z_{+} > z_{-}$, and assume a set of $N$ 3D lines
$L_i$, none of which is horizontal. Specify the lines by their
intersections with the planes of heights $z_\pm$ at points
$(x_{+}^{i},y_{+}^{i}, z_+)$ and $(x_{-}^{i},y_{-}^{i}, z_-)$, for
$i=1\ldots N$.

Define
$$x^i(\lambda) = \lambda x^i_+ + (1-\lambda) x^i_-$$
$$y^i(\lambda) = \lambda y^i_+ + (1-\lambda) y^i_-$$
$$z(\lambda) = \lambda z_- + (1-\lambda) z_-$$ For any $\lambda$,
$(x^i(\lambda), y^i(\lambda), z(\lambda))$ is a point on line $L_i$;
whether $0\le\lambda\le 1$ determines whether the point is an
interpolation between or extrapolation beyond the $z_\pm$ planes (in
either case, we will just use the term interpolated). Together, all
the points $(x^i(\lambda), y^i(\lambda), z(\lambda))$ for $i=1\ldots
N$ represent the intersection of lines $L_i$ with the plane with
height $z(\lambda)$.

Note that the set of all means
$(\bar{x}(\lambda),\bar{y}(\lambda),z(\lambda))$ comprise a line. For
the plane at height $z(\lambda)$ that yields the smallest spread of
points $(x^i(\lambda),y^i(\lambda)$ will be the point on that line
which we choose as our answer.

The two-dimensional spread of the intersection set at a particular
$z(\lambda)$ is a symmetric, positive definite, 2x2
covariance matrix

\[M(\lambda) =
\begin{bmatrix}
 var(x^i(\lambda)) && covar(x^i(\lambda),y^i(\lambda)) \\
  -                && var(y^i(\lambda))
\end{bmatrix}
\]

It can be shown that the covariance of the interpolalated points
$(x^i(\lambda),y^i(\lambda))$ can be expressed in terms of variances of and covariances between the four elementary datasets $x^i_+, x^i_-, y^i_+, y^i_-$ as follows:

\begin{equation}\label{varx}
var(x^i(\lambda)) = \lambda^2\sigma^2_{x+x+} + \lambda(1-\lambda)(\sigma^2_{x+x-} + \sigma^2_{x-x+}) 
                + (1-\lambda)^2\sigma^2_{x-x-}
\end{equation}
\begin{equation}\label{vary}
var(y^i(\lambda)) = \lambda^2\sigma^2_{y+y+} + \lambda(1-\lambda)(\sigma^2_{y+y-} + \sigma^2_{y-y+}) 
                + (1-\lambda)^2\sigma^2_{y-y-}\end{equation}
\begin{equation}\label{covarxy}
covar(x^i(\lambda),y^i(\lambda)) = \lambda^2\sigma^2_{x+y+}  
                        + \lambda(1-\lambda)(\sigma^2_{x+y-} + \sigma^2_{x-y+}) 
                              + (1-\lambda)^2\sigma^2_{x-y-}
\end{equation}

where
$$\sigma^2_{x+x+}=var(x^i_+),$$
$$\sigma^2_{x+y-}=covar(x^i_+, y^i_-),$$
etc. Note in particular that, since all the $\sigma^2$ are functions only of constants $x^i_\pm, y^i_\pm$, and $N$, each of the 4 terms of $M(\lambda)$ is a quadratic function of $\lambda$.

If we denote the determinant of a matrix with $|\cdot|$, the area of the
1-sigma error ellipse of $M(\lambda)$ is
$$a(\lambda) = \pi\sqrt{|M(\lambda)|},$$
i.e. the square root of a quartic (4th degree) polynomial.

We seek to minimize $a(\lambda)$, which is equivalent to minimizing
the quartic polynomial 
\begin{equation*}
d(\lambda) = |M(\lambda)| \\
           = var(x^i(\lambda))var(y^i(\lambda)) - covar(x^i(\lambda),y^i(\lambda))^2
\end{equation*}

Note that, for efficient computation, the quadratic, linear, and
constant coefficients of equations (\ref{varx}-\ref{covarxy}) can be
computed to yield

\begin{equation}\label{qx}
var(x^i(\lambda)) = a_x\lambda^2 + b_x\lambda + c_x
\end{equation}
\begin{equation}\label{qy}
var(y^i(\lambda)) = a_y\lambda^2 + b_y\lambda + c_y
\end{equation}
\begin{equation}\label{qxy}
covar(x^i(\lambda),y^i(\lambda)) = a_{xy}\lambda^2 + b_{xy}\lambda + c_{xy}
\end{equation}

Given coefficients computed in equations (\ref{qx}-\ref{qxy}),
$d(\lambda)$ can then be expressed as
\begin{equation*}\label{d}
d(\lambda) = (a_x\lambda^2 + b_x\lambda + c_x)(a_y\lambda^2 + b_y\lambda + c_y) - 
(a_{xy}\lambda^2 + b_{xy}\lambda + c_{xy})^2
\end{equation*}
\begin{equation}
= (a_xa_y-a_{xy}^2)\lambda^4 + (\ldots)\lambda^3 + (\ldots)\lambda^2 + (\ldots)\lambda + (\ldots)
\end{equation}

The value $\lambda_{min}$ which yields the minimum value of
$d(\lambda)$ will also determine the height $z(\lambda_{min})$ of the
output ground point, and then $\lambda_{min}$ can be used to
interpolate the intersection set
$(x^i(\lambda_{min}),y^i(\lambda_{min}))$, of which the mean
$(\bar{x}(\lambda_{min}),\bar{y}(\lambda_{min}))$ constitutes the
horizontal component of the output ground point.

\subsubsection{Non-uniqueness}
It is desirable that this quartic polynomial $d(\lambda)$ have no
local minima, but only a single, global minimum. Equivalently, the
cubic derivative should have a single real root and two complex
roots. Unfortunately, there can be degenerate arrangements of image
rays with three real roots of $d'(\lambda$ and multiple minima for
$d(\lambda)$.

Consider the bimodal situation depicted in Fig. \ref{bimodal_bundle}:
images of the ground point are captured from satellite positions
spaced equally around a horizontal circle, with orientations perfectly
intersecting at a ground point at height $z_+$. To this bundle add a
duplicate bundle, shifted both horizontally and downwards, to
intersect at a lower height $z_-$. Clearly, the spread of the joint
bundle at heights $z(\lambda=0)=z_-$ and $z(\lambda=1)=z_+$ are the
same. And the spread at $z(\lambda=1/2)$ is somewhat larger (note that
the shape of $d(\lambda)$ is depicted sideways to the right). Thus
$\lambda={0,1}$ present two minima of $d(\lambda)$, and the
hourglassing procedure in this case cannot provide a clear answer.

At least we can compute the exact form of $d(\lambda)$, and with
standard techniques understand clearly whether such a degenerate
situation were ever to present itself. (TBD: In our empirical testing
in section \ref{simulation}, we [never/seldom] encountered such a
degenerate case, in NNNN hourglassing computations involving from 4 to
1000 images.)

\subsubsection{Estimated horizontal error}
Following section \ref{MIG}, we seek to provide a meaningful estimate
of error that is informed by the central limit theorem's general
principle of uncertainty that decreases by a factor of $\sqrt{N}$, and
thus grows arbitrarily small as the number of images $N$ increases.

We chose as horizontal location for our point, the mean
$(\bar{x}(\lambda_{min}),\bar{y}(\lambda_{min}))$ of spread-minimizing
$\lambda_{min}$. If we consider the true answer $(\hat{x},\hat{y})$ to
be the goal we are attempting to measure with a sample of $N$ possible
imaging rays out of an infinity of possibilities, then we intuit that
the estimator $(\bar{x}(\lambda_{min}),\bar{y}(\lambda_{min}))$ should
be proportional to $\sqrt{N}a(\lambda_{min})$. In section
\ref{simulation} we evaluate this measure of accuracy using a
simulation with known ground truth (and find it's wickid awesome).

We also want to verify that our estimator for horizontal error
responds properly to the width of the ray bundle. Wider bundles should
yield larger horizontal error, and narrower bundles should yield
smaller horizontal error.

\subsubsection{Estimated vertical error}
For vertical error, an estimator for the empirical technique of
hourglassing does not readily present itself. Apart from matching
reality in simulations with ground truth, and behaving comparably to
MIG, the estimator needs to have the fundamental property that, for a
wider bundle must have a lower estimate of vertical error, and a
narrower bundle must have a higher estimate of vertical error.

Consider the (continuous? smooth?) surface formed by stacking the
1-sigma error ellipse for every height $z(\lambda)$ (for
$\lambda\in\Re$). Its intersection with a horizontal plane of height
$z(\lambda)$ is an ellipse centered on
$(\bar{x}(\lambda),\bar{y}(\lambda),z(\lambda)$, with area
$a(\lambda)$, and major and minor axes with direction and length
defined by the eigenvectors and square-roots of eigenvalues of
$M(\lambda)$. (Can we somehow generate a visualization of this from a
concrete example?)

Narrowness of this roughly conical surface corresponds to a low amount
of curvature of $a(\lambda)$, and a shallow minimum which yields
uncertainty as to which height plane is really the best
estimate. Wideness corresponds to sharp curvature of $a(\lambda)$ and
more certainty that the chosen height is close to correct.

Sharpness of curvature can be measured by a large 2nd derivative, so
we can consider various measures that increase with
$a''(\lambda_{min})$ or $d''(\lambda_{min})$.

However, we also want our estimate of vertical uncertainty to improve
with $N$, and $a''(\lambda_{min})$ by itself will simply more
accuracetly estimate the width of the whole bundle as $N$
increases. So, similar to horizontal error above, we will evaluate
combining a factor of $1/\sqrt{N}$ in our estimate to give it behavior
consistent with the central limit theorem.

\subsubsection{Weighting}
So far, our development of the hourglassing heuristic has applied no
weights, which is to say, all image rays have been weighted
equally. If some source of information provides distinct weights to
various image rays, it is possible to incorporate them.

Obviously, it is possible to compute
$\bar{x}(\lambda),\bar{y}(\lambda)$ with a weighted average, which
will pull the average closer to more heavily weighted image rays.

As for measuring spread, it is clear that if the weights are all
integral, the spread calculation could be modified by including
duplicates of each ray according to its weight, and the same method
applies. The technique extends also to fractional weights. (adjusted
formulas?)
 

\section{Results with simulated sensor models}\label{simulation}
In this section we develop a large simulated testbed, and evaluate
MIG, and hourglassing with various experimental measures for estimated
error, on samples of randomly-selected subsets of images, of size
ranging from 4 to 1000.

To develop the test set, we start with a large as possible set of
real, unadjusted sensor models which have common overlap. From
(\ref{MIN}) we have 10 Worldview-1 sensor models which all view the
ground location 36N 117.5W 1700mHAE (WGS84). This point is set as
Ground Truth.

We then extend the set of real sensor models by random
perturbation. We repeatedly add random amounts of correction to
position and orientation adjustable parameters. The random corrections
are sampled from uniform distributions with bounds many times the
nominal triangulation defaults, with the aim of spreading the
simulated sensors as uniformly as possible, minimizing clustering
around the original sensor models. (See resulting clustering in Fig
\ref{clusterfig}). Most large random perturbations of this kind will
steer the real sensor models away from viewing our truth point. Sensor
models that still contain the truth point within their image bounds
are retained, until we have 99 perturbations for each original, real
sensor model, for a total of 1000 sensor models.

The simulated set of 1000 images thus constructed is considered the
Truth set of sensor models, with position and orientation parameters
that perfectly represent conditions at image collection, to which
actual sensor models would be imperfect estimates. The sensor model's
ground to image function is used to project the truth point into image
coordinates for each sensor model. These image measurements are
retained as truth as well. The ray bundle emanating from these truth
image points intersect perfectly (to within computational precision of
the image to ground function) at the truth point in ground space. Thus
the MIG and Hourglassing approaches would both yield the ground point,
for 1000 images or for any subset.

Once the idealized bundle of 1000 image rays is assembled, experiments
can be conducted by adding a controlled amount of error (randomly
sampled from a known distribution) to each sensor model. The 1000
sensor models thus perturbed represent a possible realization of 1000
images with position and orientation parameters being different from
their actual truth. The idealized image measurements can be used
(because that is where the visualization of the ground feature
actually appears), a controlled amount of error can be added in image
space (to represent a desired amount of image measurement error, or
unmodeled sensor error). Using perturbed image measurements/sensor
models, the resulting ray bundle will represent a realistic spread of
image rays around the truth point, to which MIG and Hourglassing
algorithms can be applied and evaluated. This procedure can be carried
out for any subset of $N$ images, in fact for any number of $N$-image
samples.

Lots of tables and figures.

\section{Results with real images}
If we can get enough, like 50+.


\section{Bibliography}
\end{document}
