\documentclass[]{article}
\usepackage{amssymb, amsmath, bm, nameref, multicol}

\newcommand{\imgmeashat}{\pmb{\hat{x}_{i}}}
\newcommand{\imgmeas}{\pmb{x_{i}}}
\newcommand{\grndhat}{\pmb{\hat{X}}}
\newcommand{\grnd}{\pmb{X}}
\newcommand{\grnditer}{\pmb{X^k}}
\newcommand{\sensmeashat}{\pmb{\hat{p}_i}}
\newcommand{\sensmeas}{\pmb{p_i}}
\newcommand{\imgnu}{\pmb{\nu_{i}^x}}
\newcommand{\sensnu}{\pmb{\nu_i^p}}
\newcommand{\grndupdate}{\pmb{\Delta}}
\newcommand{\grndupdateiter}{\pmb{\Delta^k}}
\newcommand{\Fimgpartials}{\frac{\partial{\pmb{F_{i}}}}{\partial{\imgmeas}}}
\newcommand{\Fgrndpartials}{\frac{\partial{\pmb{F_{i}}}}{\partial{\grnd}}}
\newcommand{\Fsenspartials}{\frac{\partial{\pmb{F_{i}}}}{\partial{\sensmeas}}}



%opening
\title{MIG Variance}
\author{Seth Merickel}

\begin{document}

\maketitle


\section*{Introduction}


\section*{MIG Equation Derivation}
The goal is to determine ground locations that minimize the weighted sum of image measurement and sensor parameter deviations using non-linear least squares.  $\grnd = [X, Y, Z]^T$ is the ground point being estimated.  $\imgmeas = [x_{i}, y_{i}]^T$ is the measurement of the ground point in the $i$th image.  The sensor parameters for the $i$th image are $\sensmeas = [p_{i1}, p_{i2}, \ldots]^T$. The ground point, image measurements, and sensor parameters are related through the sensor's ground to image function $\imgmeas = \pmb{g_i}(\sensmeas, \grnd)$.  Let $\pmb{F_{i}}$ be the measurement residual function which is the difference between the projected ground point and image measurement

\begin{equation*}
\pmb{F_{i}}(\sensmeashat, \grndhat,\imgmeashat)=\pmb{g_i}(\sensmeashat,\grndhat)-\imgmeashat
\end{equation*}

Where $\imgmeashat$, $\sensmeashat$, and $\grndhat$ are measured quantities related to the predicted quantities by:

\begin{equation*}
\begin{split}
\imgmeashat = \imgmeas + \imgnu\\
\sensmeashat    = \sensmeas + \sensnu\\
\grndhat    = \grnd + \grndupdate
\end{split}
\end{equation*}

$\imgnu$ and $\sensnu$ are differences in the predicted and measured image coordinates and sensor parameters respectively.  $\grndupdate$ is the correction to the ground point.

Taylor expanding $\pmb{F_{i}}$ about the predicted values gives

\begin{equation} \label{taylor_eq}
\pmb{F_{i}}(\sensmeashat, \grndhat, \imgmeashat) = 
\pmb{F_{i}}(\sensmeas, \grnd, \imgmeas) + \Fimgpartials\imgnu + \Fsenspartials\sensnu + \Fgrndpartials\grndupdate = 0
\end{equation}

Sensor parameter partials are denoted by $A_{i}^p = \Fsenspartials$, and the sensor ground partials by $B_{i} = \Fgrndpartials$.  The partials with respect to image coordinates are ${\Fimgpartials=-I_{2 \times 2}}$, the negative of the $2 \times 2$ identity matrix.  Substituting $A_{i}^p$ and $B_{i}$ into (\ref{taylor_eq}) and rearranging terms gives:

\begin{equation} \label{linear_eq}
\begin{split}
-\imgnu + A_{i}^p\sensnu + B_{i}\grndupdate = \imgmeas - \pmb{g_i}(\sensmeas, \grnd) = \pmb{f_{i}}
\end{split}
\end{equation}

Stacking up equations for each measurement into one vector equation gives

\begin{equation}\label{stacked_eq}
-\pmb{\nu^x} + A^p\pmb{\nu^p} + B\grndupdate = \pmb{f}
\end{equation}

The function to be minimized is the weighted sum of the measurement deviations and sensor deviations given by $\Phi = \pmb{\nu^x}^T W^x \pmb{\nu^x} + \pmb{\nu^p}^T W^p \pmb{\nu^p}$ subject to the constraints in \ref{stacked_eq}.  $W^x$ and $W^p$ are inverses of the measurement covariance matrix and sensor parameter covariance matrix respectively  For the details of this procedure see the discussion of the Gauss-Helmert model in (add references to Photogrammetric Computer Vision and Manual of Photogrammetry).  The resulting normal equations are:

\begin{equation} \label{normal_eq}
B^T W B\pmb{\Delta} = B^T W \pmb{f}
\end{equation}

where the weight matrix $W$ is:

\begin{equation}\label{weight_eq}
W = ({W^x}^{-1} + A^p {W^p}^{-1} {A^p}^T)^{-1}
\end{equation}

This is a non-linear problem so iteration is used to converge to a solution.  After each iteration the ground point locations from the previous iteration are updated $\pmb{X^k} = \pmb{X^{k - 1}} + \pmb{\Delta^k}$  and the partial derivative matrices $A$ and $B$ are evaluated at the new ground point location for the next iteration.  Convergence is typically achieved after only a few iterations.  

\section*{Matrices}
Understanding the form of the partial derivative matrices is important for analyzing how the ground point estimate covariance depends on the number of images.  For the $i$th measurement the ground and sensor partials are

\begin{align}\label{ith_partials}
\begin{aligned}
A_{i} = 
\begin{bmatrix}
\frac{\partial{\pmb{g_i}}}{\partial{p_{i1}}} & \frac{\partial{\pmb{g_i}}}{\partial{p_{i2}}} & \ldots
\end{bmatrix}  
&& 
B_{i} =
\begin{bmatrix}
\frac{\partial{\pmb{g_i}}}{\partial{X}} & \frac{\partial{\pmb{g_i}}}{\partial{Y}} & \frac{\partial{\pmb{g_i}}}{\partial{Z}}
\end{bmatrix}
\end{aligned}
\end{align}

Then $A$ and $B$ in normal equations (\ref{normal_eq}) have the form:

\begin{align}\label{full_partials}
\begin{aligned}
A = 
\begin{bmatrix}
A_{1} &  & \\
& 	A_{2} \\
&   & \ddots
\end{bmatrix}
&& 
B = 
\begin{bmatrix}
B_{1}  \\
B_{2} \\
\vdots
\end{bmatrix}
\end{aligned}
\end{align}

\end{document}
