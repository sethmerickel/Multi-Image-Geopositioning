\documentclass{amsart}
%\documentclass{article}
%\usepackage{amsmath}
\usepackage{graphicx} % for png figures

\begin{document}

\newcommand{\Iimg}{\mathcal{I}}
\newcommand{\Pimg}{\mathcal{P}}

\title{Hourglassin'}
\author{Rube}
\author{Seth}
\begin{abstract}
Abstract goes here
\end{abstract}
\maketitle

\section{Introduction}
We be MIG'n and hourglassin'.

\section{Least-Squares Multi-Image Geopositioning With Many Images}
\subsection{MIG}\label{MIG}
Use least-squares to show
$\lim_{n\rightarrow\inf}err=0$.

\subsection{Simulation with Synthetic Imagery}
In this section we develop a large simulated testbed, and evaluate Least-squares
MIG with various experimental measures for estimated error, on samples of
randomly-selected subsets of images, of size ranging from 4 to 1000.

To develop the test set, we start with a large as possible set of real,
unadjusted sensor models which have common overlap. From \cite{MIN} we have 10
Worldview-1 sensor models which all view the ground location 36N 117.5W 1700mHAE
(WGS84). This point is set as Ground Truth.

We then extend the set of real sensor models by random perturbation. We
repeatedly add random amounts of correction to position and orientation
adjustable parameters. The random corrections are sampled from uniform
distributions with bounds many times the nominal triangulation defaults, with
the aim of spreading the simulated sensors as uniformly as possible, minimizing
clustering around the original sensor models. (See resulting clustering in Fig
\ref{clusterfig}). Most large random perturbations of this kind will steer the
real sensor models away from viewing our truth point. Sensor models that still
contain the truth point within their image bounds are retained, until we have 99
perturbations for each original, real sensor model, for a total of 1000 sensor
models.

The simulated set of 1000 images thus constructed is considered the Truth set of
sensor models, with position and orientation parameters that perfectly represent
conditions at image collection, to which actual sensor models would be imperfect
estimates. The sensor model's ground to image function is used to project the
truth point into image coordinates for each sensor model. These image
measurements are retained as truth as well. The ray bundle emanating from these
truth image points intersect perfectly (to within computational precision of the
image to ground function) at the truth point in ground space. Thus MIG would
yield the truth ground point, for 1000 images or for any subset. Call this
idealized set of images $\Iimg$.

Once $\Iimg$ is assembled, experiments can be conducted by adding a controlled
amount of error (randomly sampled from a known distribution) to each sensor
model. The 1000 sensor models thus perturbed represent a possible realization of
1000 images with position and orientation parameters being different from their
actual truth. The idealized image measurements can be used (because that is
where the visualization of the ground feature actually appears), a controlled
amount of error can be added in image space (to represent a desired amount of
image measurement error, or unmodeled sensor error). Using perturbed image
measurements/sensor models, the resulting ray bundle will represent a realistic
spread of image rays around the truth point, to which MIG and Hourglassing
algorithms can be applied and evaluated. This procedure can be carried out for
any subset of $N$ images, in fact for any number of $N$-image samples.

Let $\Pimg$ be the set of 1000 perturbed images. In this experiment, for each
$N\in{4,5,...100,105,110,...995,1000}$, subsets of $N$ images are sampled from
the $\Pimg$ perturbed images $k=100$ times (except for $N=1000$, where the
entire set is used only $k=1$ times). MIG is used to estimate the geolocation of
the point using the $N$-image subset. Because we know the truth ground point,
the trivial intersection of $\Iimg$, we can compute the actual error of each of
these MIGs.

Figure \ref{fig:vanillaref} shows the reference variances of every MIG from this
experiment. Reference variance is tightly clustered around 1, which demonstrates
that the use of apriori covariance in the MIG calculations is consistent with
the errors involved in the perturbation of $\Pimg$.

Figure \ref{fig:vanillaxyz} shows the average of errors in the X, Y, and Z
dimensions. Each point represents an average over $k=100$ MIGS. Because errors
in positive and negative directions cancel each other out in these averages,
this graph shows the amount of bias in the MIG calculations across
$N$-subsets. For the full set $\Pimg$ ($N=1000$), $dX\approx 0$, $dY\approx
-3$cm, and $dZ\approx 5$cm. This is the overall error that perturbation
introduced into $\Pimg$, and it is clear that the average MIG converges to this,
showing that the MIG is not a biased estimate.

To avoid positive and negative errors cancelling each other out, we switch to
positive quantities. Figure \ref{fig:vanilla_pred_meas} shows the average
predicted and measured CE90/LE90 for each $N$ across the $k=100$ MIGs for that
$N$. To clarify the rate of convergence and separation of the curves, a
logarithmic scale is used. (The measured errors are adjusted with a finite
population correction of $\sqrt{(1000-N)/(1000-1)}$.)

Figure \ref{fig:vanilla_pred_meas_sqrtn} multiplies each point from Figure
\ref{fig:vanilla_pred_meas} by a factor of $\sqrt N$. The flatness of these
curves demonstrates the thesis of this paper, that MIG follows the Law of Large
Numbers, and exhibits error that decreases as $1/\sqrt N$. (The upturn of
measured error near $N=1000$ is a result of the limitations of the finite
population correction near the total finite population.)


\begin{figure}
\includegraphics[width=\textwidth]{fig_refvar.png}
\caption{\label{fig:vanillaref}All reference variances}
\end{figure}

\begin{figure}
\includegraphics[width=\textwidth]{fig_avgxyz.png}
\caption{\label{fig:vanillaxyz}Average Errors in X, Y, and Z}
\end{figure}

\begin{figure}
\includegraphics[width=\textwidth]{fig_pred_meas.png}
\caption{\label{fig:vanilla_pred_meas}Pred/Meas CE90/LE90 (log scale)}
\end{figure}

\begin{figure}
\includegraphics[width=\textwidth]{fig_pred_meas_sqrtn.png}
\caption{\label{fig:vanilla_pred_meas_sqrtn}Pred/Meas CE90/LE90, multiplied by $\sqrt N$}
\end{figure}


\section{Hourglassin'}
\subsection{Motivation}
Because available sensor models (such as RPC) often lack a meaningful
error model to input to MIG, and for computational simplicity, we
introduce a heuristic approach as an alternative to rigorous
least-squares MIG. Although when visualized 'up close' to the true
answer, any particular tangle of rays may seem unclear about where the
ground point should be localized, from a global perspective, the ray
bundle will be something like a cone, widening to the cluster of
satellite (or airborne) perspective centers, narrowing at the ground
point, and widening again beyond the ground point. We seek the answer
at the narrowest point of the cone. In an ideal case, the ray bundle
will intersect at a single point, which has cross-sectional area of
0. In a real-world case, the bundle will appear as a cone with a `fat'
intersection. Thus the name `hourglassing' to motivate the heuristic
technique.

Given a bundle of rays, we can intersect the bundle with planes of
various heights, compute the collection of intersections of the ray
bundle with each height plane, measure the spread of the 2-D
distribution of points, and choose the plane with the least spread to
be the solution for the height of the desired ground point. For the
horizontal location of the ground point, the natural choice is the
mean of the intersection points in the chosen plane.

Rather than attacking this problem with a brute force search for the
spread-minimizing by computing intersection sets at very many heights,
and slicing height space sufficiently thin to achieve a desired
vertical resolution, we attempt some theoretical underpinnings for
this technique that will allow a more efficient and precise solution,
and help motivate a meaningful estimate of the error of the resulting
ground point.

\subsection{Computing height of minimum spread}
Assume two heights $z_{+} > z_{-}$, and assume a set of $N$ 3D lines
$L_i$, none of which is horizontal. Specify the lines by their
intersections with the planes of heights $z_\pm$ at points
$(x_{+}^{i},y_{+}^{i}, z_+)$ and $(x_{-}^{i},y_{-}^{i}, z_-)$, for
$i=1\ldots N$.

Define
$$x^i(\lambda) = \lambda x^i_+ + (1-\lambda) x^i_-$$
$$y^i(\lambda) = \lambda y^i_+ + (1-\lambda) y^i_-$$
$$z(\lambda) = \lambda z_- + (1-\lambda) z_-$$ For any $\lambda$,
$(x^i(\lambda), y^i(\lambda), z(\lambda))$ is a point on line $L_i$;
whether $0\le\lambda\le 1$ determines whether the point is an
interpolation between or extrapolation beyond the $z_\pm$ planes (in
either case, we will just use the term interpolated). Together, all
the points $(x^i(\lambda), y^i(\lambda), z(\lambda))$ for $i=1\ldots
N$ represent the intersection of lines $L_i$ with the plane with
height $z(\lambda)$.

Note that the set of all means
$(\bar{x}(\lambda),\bar{y}(\lambda),z(\lambda))$ comprise a line. For
the plane at height $z(\lambda)$ that yields the smallest spread of
points $(x^i(\lambda),y^i(\lambda)$ will be the point on that line
which we choose as our answer.

The two-dimensional spread of the intersection set at a particular
$z(\lambda)$ is a symmetric, positive definite, 2x2
covariance matrix

\[M(\lambda) =
\begin{bmatrix}
 var(x^i(\lambda)) && covar(x^i(\lambda),y^i(\lambda)) \\
  -                && var(y^i(\lambda))
\end{bmatrix}
\]

It can be shown that the covariance of the interpolalated points
$(x^i(\lambda),y^i(\lambda))$ can be expressed in terms of variances of and covariances between the four elementary datasets $x^i_+, x^i_-, y^i_+, y^i_-$ as follows:

\begin{equation}\label{varx}
var(x^i(\lambda)) = \lambda^2\sigma^2_{x+x+} + \lambda(1-\lambda)(\sigma^2_{x+x-} + \sigma^2_{x-x+}) 
                + (1-\lambda)^2\sigma^2_{x-x-}
\end{equation}
\begin{equation}\label{vary}
var(y^i(\lambda)) = \lambda^2\sigma^2_{y+y+} + \lambda(1-\lambda)(\sigma^2_{y+y-} + \sigma^2_{y-y+}) 
                + (1-\lambda)^2\sigma^2_{y-y-}\end{equation}
\begin{equation}\label{covarxy}
covar(x^i(\lambda),y^i(\lambda)) = \lambda^2\sigma^2_{x+y+}  
                        + \lambda(1-\lambda)(\sigma^2_{x+y-} + \sigma^2_{x-y+}) 
                              + (1-\lambda)^2\sigma^2_{x-y-}
\end{equation}

where
$$\sigma^2_{x+x+}=var(x^i_+),$$
$$\sigma^2_{x+y-}=covar(x^i_+, y^i_-),$$
etc. Note in particular that, since all the $\sigma^2$ are functions only of constants $x^i_\pm, y^i_\pm$, and $N$, each of the 4 terms of $M(\lambda)$ is a quadratic function of $\lambda$.

If we denote the determinant of a matrix with $|\cdot|$, the area of the
1-sigma error ellipse of $M(\lambda)$ is
$$a(\lambda) = \pi\sqrt{|M(\lambda)|},$$
i.e. the square root of a quartic (4th degree) polynomial.

We seek to minimize $a(\lambda)$, which is equivalent to minimizing
the quartic polynomial 
\begin{equation*}
d(\lambda) = |M(\lambda)| \\
           = var(x^i(\lambda))var(y^i(\lambda)) - covar(x^i(\lambda),y^i(\lambda))^2
\end{equation*}

Note that, for efficient computation, the quadratic, linear, and
constant coefficients of equations (\ref{varx}-\ref{covarxy}) can be
computed to yield

\begin{equation}\label{qx}
var(x^i(\lambda)) = a_x\lambda^2 + b_x\lambda + c_x
\end{equation}
\begin{equation}\label{qy}
var(y^i(\lambda)) = a_y\lambda^2 + b_y\lambda + c_y
\end{equation}
\begin{equation}\label{qxy}
covar(x^i(\lambda),y^i(\lambda)) = a_{xy}\lambda^2 + b_{xy}\lambda + c_{xy}
\end{equation}

Given coefficients computed in equations (\ref{qx}-\ref{qxy}),
$d(\lambda)$ can then be expressed as
\begin{equation*}\label{d}
d(\lambda) = (a_x\lambda^2 + b_x\lambda + c_x)(a_y\lambda^2 + b_y\lambda + c_y) - 
(a_{xy}\lambda^2 + b_{xy}\lambda + c_{xy})^2
\end{equation*}
\begin{equation}
= (a_xa_y-a_{xy}^2)\lambda^4 + (\ldots)\lambda^3 + (\ldots)\lambda^2 + (\ldots)\lambda + (\ldots)
\end{equation}

The value $\lambda_{min}$ which yields the minimum value of
$d(\lambda)$ will also determine the height $z(\lambda_{min})$ of the
output ground point, and then $\lambda_{min}$ can be used to
interpolate the intersection set
$(x^i(\lambda_{min}),y^i(\lambda_{min}))$, of which the mean
$(\bar{x}(\lambda_{min}),\bar{y}(\lambda_{min}))$ constitutes the
horizontal component of the output ground point.

\subsection{Empirical Equivalence with MIG}
For every MIG calculation that went into Figures
\ref{fig:vanillaref}--\ref{fig:vanilla_pred_meas_sqrtn}, an Hourglass geolocation
calculation was also performed. Figure \ref{fig:mig_vs_hourglass} shows the
actual errors in the X, Y, and Z directions from a MIG calculation (horizontal
axis) versus an Hourglass calculation (vertical axis), for all of those
calculations. (Every point in Figure \ref{fig:vanillaxyz} corresponds with
$k=100$ points in Figure \ref{fig:mig_vs_hourglass}.) The scatter follows the
unity line $y=x$ very closely, showing that Hourglassing yields very closely the
same geolocation as MIG. (TBD: restricted graph for N>10 or N>20, showing
outliers are gone?)

\begin{figure}
\includegraphics[width=\textwidth]{fig_mig_ply_xyz_sma.png}
\includegraphics[width=\textwidth]{fig_mig_ply_xyz_big.png}
\caption{\label{fig:mig_vs_hourglass}MIG vs Hourglass errors.  Above for $N\le
  10$, below for $N>10$. Each triplet is one graph each for X, Y, and Z errors.}
\end{figure}


\subsection{Non-uniqueness}
It is desirable that this quartic polynomial $d(\lambda)$ have no
local minima, but only a single, global minimum. Equivalently, the
cubic derivative should have a single real root and two complex
roots. Unfortunately, there can be degenerate arrangements of image
rays with three real roots of $d'(\lambda$ and multiple minima for
$d(\lambda)$.

Consider the bimodal situation depicted in Fig. \ref{bimodal_bundle}:
images of the ground point are captured from satellite positions
spaced equally around a horizontal circle, with orientations perfectly
intersecting at a ground point at height $z_+$. To this bundle add a
duplicate bundle, shifted both horizontally and downwards, to
intersect at a lower height $z_-$. Clearly, the spread of the joint
bundle at heights $z(\lambda=0)=z_-$ and $z(\lambda=1)=z_+$ are the
same. And the spread at $z(\lambda=1/2)$ is somewhat larger (note that
the shape of $d(\lambda)$ is depicted sideways to the right). Thus
$\lambda={0,1}$ present two minima of $d(\lambda)$, and the
hourglassing procedure in this case cannot provide a clear answer.

At least we can compute the exact form of $d(\lambda)$, and with
standard techniques understand clearly whether such a degenerate
situation were ever to present itself. (TBD: In our empirical testing
in section \ref{simulation}, we [never/seldom] encountered such a
degenerate case, in NNNN hourglassing computations involving from 4 to
1000 images.)


\subsection{Error Estimation}
Review of Figures \ref{fig:vanilla_pred_meas}--\ref{fig:vanilla_pred_meas_sqrtn}
suggests an empirical method of error estimation for Hourglassing. Predicted
error decreases very predictably as $1/\sqrt N$. So an alternative method for
estimating the error of a MIG with $N$ images would be to subsample $M<<N$
images many times, compute the sample covariance $C_M$ of the $M$-MIGs, and
because of the $1/\sqrt N$ behavior (which is $1/N$ in variance space), predict
the error of the $N$-MIG to be $$C_N=\frac{C_M}{N/M}$$.

This allows estimation of error apart from the output covariance of the $N$-MIG
(or even of any of the $M$-MIG). There's not much point to this for MIG, since
output covariance is a natural by-product of the MIG algorithm. But for
Hourglassing, this provides a method of error estimation which requires no error
information for the bundle, simply calculation of additional, smaller Hourglass
calculations on many subamples, and computation and scaling of the sample
covariance of the resulting set of geolocations.

The fundamental principle behind this error estimation technique, is that
apriori error information is not needed for each image in the bundle, because
the bundle {\em is} the distribution of error. For typical collections of only 2
or 4 images, the sample size is too small to reliably know whether the ray
separation is a true representation of the amount of error in the system, or
whether it may be concidentally large (or small). Thus a traditional MIG with
apriori and aposteriori covariance, and reference variance as a consistency
check, is most appropriate. But for large enough collections of images, the
distribution of image rays is a reliable summary of the error of the process.

\subsection{Error Estimation Simulated Results}
To evaluate this empirical method of error estimation for Hourglassing, a subset
of $N=100$ images from $\Pimg$ were chosen, and an Hourglass-geolocation
computed for them. Subsets of $M\in\{25,50,75\}$ were sampled from the $N$-set,
$k\in\{100,200,400\}$ times, and an Hourglass-geolocation is computed for each
$M$-subset. 3x3 sample covariance was computed for those $k$ ground points, and
then scaled by $1/(N/M)=M/N$ because of the $1/\sqrt{N}$ effect, and corrected
also by a FPC factor of $(N-1)/(N-M)$, to yield the covariance estimate for the
$N$-Hourglass. A MIG was also computed for the same $N$-subset, and its output
covariance computed for reference. This experiment was repeated $R=100$ times.

Figure \ref{fig:mig_vs_hourglass_var} shows the variance of X from MIG vs
Hourglass error estimation. The Hourglass error estimation has more spread than MIG.

\begin{figure}
\includegraphics[width=\textwidth]{fig_self_var_xy.png}
\includegraphics[width=\textwidth]{fig_self_var_z.png}
\caption{\label{fig:mig_vs_hourglass_var}MIG vs Hourglass estimated variances}
\end{figure}

\section{Conclusion}
In conclusion, we conclude.





\begin{thebibliography}{9} % so far, only 1-digit reference nums

\bibitem{MIN}Dolloff, John, and Reuben Settergren, ``Worldview-1 Stereo
  Extraction Accuracy With and Without MIN Processing.'' ASPRS 2010.

\bibitem{LLN}Law of large numbers citation, my stats book on the shelf at work

\bibitem{FPC}Finite Population Correction reference, look up something

\bibitem{LSQRMIG}Mikhail/Bethel/McGlone? Manual of Photogrammetry?

\end{thebibliography}
 
\end{document}
