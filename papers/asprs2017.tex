\documentclass[10pt]{amsart}
\usepackage[total={6.5in,9in}]{geometry}
%\documentclass{article}
%\usepackage{amsmath}
\usepackage{graphicx} % for png figures
\usepackage{subcaption} % for subfigures
\usepackage{amsaddr} % addresses up front
\pagenumbering{gobble} % no page numbers


\begin{document}

\newcommand{\Iimg}{\mathcal{I}}
\newcommand{\Pimg}{\mathcal{P}}

\title[Small Geopositioning Error from Many Images]{}
%{Asymptotically Small Geopositioning Error from Large Sets of Imagery}
\author[Settergren]{}%{\bf \rm{Reuben Settergren}}
\author[Merickel]{}%{Seth Merickel}
\maketitle


\vskip -1.1in

\begin{centering}
{\large \bf ASYMPTOTICALLY SMALL GEOPOSITIONING ERROR \\ FROM LARGE SETS OF IMAGERY}

\vspace \baselineskip

{\bf Reuben Settergren}

{\bf Seth Merickel}

{\tt rjs@jhu.edu}

{\tt sethmerickel@gmail.com}

\end{centering}

\vspace \baselineskip

%\begin{abstract}
%\begin{quotation}

{\noindent \large \bf ABSTRACT}
\vspace \baselineskip

\noindent The geolocation error resulting from least-squares Multi-Image Geopositioning
(MIG) with $N$ images is shown theoretically and empirically to decrease as
$1/\sqrt{N}$, using a synthetically augmented set of 1000 overlapping
Worldview-1 images. A novel heuristic approach called `Hourglassing' is
introduced, which requires no understanding of apriori covariance, but
implicitly infers uncertainty from the distribution of the ray bundle
itself. Hourglassing geolocation and error estimation is compared to MIG using
the 1000-image testbed.
%\end{quotation}
%\end{abstract}

\section{Introduction}
Least-squares Multi-Image Geopositioning (hereafter MIG) is well-under\-stood in
photogrammetry\cite{LSQRMIG}, providing geolocation and output covariance that
are as accurate as the apriori projection and covariance data that are input
into the process. However, for practical reasons, the number of images that have
been used in MIG calculations have been rather small. Even after the advent of
computing enabled the automatic processing of many images, there still remained
the practical problem of {\em having} very many images that observed the same
ground feature (as well as meaningful apriori covariance models). A typical
airborne collection with a mapping camera and 60/40 forward/side-lap
configuration would yield at most 6 views. A collection with large satellite
images is usually designed to have minimally sufficient overlap, yielding only
1- or 2-deep coverage almost everywhere (depending on a mono or stereo collect),
with imagery being 2- or 4-deep at the overlaps only to ensure there are no gaps.

Some authors have applied MIG with more extensive collections of imagery. In
\cite{JEONG_SIX}, Jeong, Yang, and Kim analyze MIG with one pair of imagery from
each of three different satellites: IKONOS, QuickBird, and KOMPSAT-2. In
\cite{PLANE_COLLINS}, R.~T.~Collins geopositions with 7 images simultaneously
(although not with MIG, but a technique called `Space-Sweep' -- which is
similar to the Hourglassing method introduced in section \ref{hourglassing}).
In \cite{LRO_NINE}, a team from the Chinese Academy of Sciences investigates MIG
accuracy with 2--9 images of the moon from the Lunar Reconnaissance
Orbiter. Wonnacott \cite{WONNACOT_32_3} analyzes MIG accuracy with SAR, using 32
image `sets' (each set being a `three-aperture path sequence').

These image collections, however, represent only a fraction of what is currently
possible, given the ubiquity of high-resolution commercial satellite imaging
from companies like Digital Globe, Airbus, SPOT, Pl\'eiades, KOMPSAT,
COSMO-SkyMed, TerraSAR-X, etc. A search of the Digital Globe online imagery
catalog \cite{BROWSE_DG} reveals well over 100 images for many large cities. And
the current wave of SmallSat flocks (Planet Labs, RapidEye, etc.) with their
rapid revisit times, promise to provide floods of repeat imagery. Although
SmallSats may be inferior in resolution and accuracy at the moment, technology
will improve, and it is the goal of this paper to show that these deficiencies
can be overcome by the power of large amounts of statistically independent
information from huge numbers of images.

\section{Least-Squares MIG With Many Images}
\subsection{MIG}\label{MIG}
TBD: Use least-squares to show $\lim_{n\rightarrow\inf}err=0$.

\subsection{Simulation with Synthetic Imagery}
In this section we construct a large testbed of simulated/\-syn\-the\-tic imagery, and
evaluate the accuracy of MIG geolocation and error estimation, on samples of
randomly-selected subsets of images, of size ranging from 4 to 1000.

To develop the test set, we start with a large as possible set of real sensor
models which have common overlap. (Note `sensor models', not `images.' For
this simulated experiment, no pixels are needed, only sensor models. Actual
scene content is irrelevant.)  From \cite{MIN} we have sensor models from 10
Worldview-1 images which all view the ground location 36N 117.5W 1700mHAE
(WGS84). This point is set as Ground Truth. (The sensor model used in this study
is the SOCET GXP \cite{SGXP} Worldview sensor model.) We then synthetically
extend the set of real sensor models by randomly perturbation position and
orientation adjustable parameters with very large corrections. Using this random
perturbation technique, we generate 99 perturbations for each of the 10
originals, for a total of 1000 sensor models.

The simulated set of 1000 images thus constructed is considered the Truth set of
sensor models. The sensor model's ground to image function is used to project
the truth point into image coordinates for each sensor model. These image
measurements are retained as truth also, and used for all the following
experimental ray-intersections. The ray bundle emanating from these truth image
points intersect perfectly (to within computational precision of the image to
ground function) at the truth point in ground space. Let this idealized set of
images and measurements be $\Iimg$.

Once the ideal bundle $\Iimg$ is assembled, an image bundle with realistic
errors can be constructed by perturbing sensor models from $\Iimg$ with a
controlled amount of error (randomly sampled from a known distribution), to
obtain perturbed image set $\Pimg$. In \cite{MIN}, the error model for the SOCET
GXP Worldview sensor model is tuned to ground truth, and adjusts 6 parameters: 3
attitude parameters at $\sigma=5\mu$rad, and 3 attitude rate parameters at
$\sigma=0.5\mu$rad/s, which yields an error of about CE90=??. For this study we
use the same sensor model, but adopt a significantly larger error model, adding
3 second-order attiude errors at $\sigma=0.05\mu\mathrm{rad}^2/\mathrm{s}^2$,
and 9 position/velocity/ac\-cel\-eration errors with $\sigma = 1{\mathrm m},$
$0.1{\mathrm m}/{\mathrm s},$ $0.01{\mathrm m}^2/{\mathrm s}^2$. (This model
yields about CE90=??? of error per image). Any subset of images of $\Pimg$
(along with the truth measurements on those images) can be used as a realistic
input to least-squares MIG. Apriori covariance input to the MIG is the same
error model that was used to perturb $\Iimg$ into $\Pimg$.

Let $\Pimg$ be the set of 1000 perturbed images. For each
$N\in\{4,5,...$ $100,105,...995\}$, we randomly sample $k=100$ subsets of size
$N$ images from $\Pimg$. Least-squares MIG is used to estimate the geolocation
of the truth point using each $N$-image subset. Because we know the truth ground
point, the trivial intersection of $\Iimg$, we can compute the actual error of
each of these MIGs, and observe the distribution of error as $N$ increases.

\begin{figure}
\includegraphics[width=.7\textwidth]{fig_refvar.png}
\caption{\label{fig:vanillaref}All reference variances. For every $N$, each of
  the $k=100$ MIG generates one pixel in the graph.}
\end{figure}

\begin{figure}
\includegraphics[width=.7\textwidth]{fig_avgxyz.png}
\caption{\label{fig:vanillaxyz}Average Errors in X, Y, and Z. The horizontal
  lines represent the MIG solution for all $N=1000$ images of $\Pimg$: (-0.0049,
  -0.0330, 0.0378), to which the subset MIGs converge.}
\end{figure}

\begin{figure}
\includegraphics[width=.7\textwidth]{fig_pred_meas.png}
\caption{\label{fig:vanilla_pred_meas}Predicted vs measured CE90/LE90 (log
  scale). Measured values are adjusted with a finite population correction.}
\end{figure}

\begin{figure}
\includegraphics[width=.7\textwidth]{fig_pred_meas_sqrtn.png}
\caption{\label{fig:vanilla_pred_meas_sqrtn}Predicted vs measured CE90/LE90,
  scaled by $\sqrt{N}$.}
\end{figure}

Figure \ref{fig:vanillaref} shows the reference variances of every MIG from this
experiment (a total of 27600 individual MIG calculations). Reference variance is
tightly clustered around 1, which demonstrates that the use of apriori
covariance in the MIG calculations is consistent with the errors involved in the
perturbation of $\Pimg$.

Figure \ref{fig:vanillaxyz} shows the average of MIG errors in the X, Y, and Z
dimensions. Each point represents an average over $k=100$ MIGS. Because errors
in positive and negative directions cancel each other out, this graph shows the
amount of bias in the MIG calculations across $N$-subsets. For all $N=1000$
images of $\Pimg$, the MIG solution differs from ground truth by (-0.0049,
-0.0330, 0.0378) meters. Horizontal lines at those values indicate that as
$N\rightarrow 1000$, bias across the $k=100$ MIGs for each $N$ converges to that
error inherent to $\Pimg$.

To avoid positive and negative errors cancelling each other out, we switch to
positive quantities. Figure \ref{fig:vanilla_pred_meas}a shows the average
predicted and measured CE90/LE90 for each $N$ across the $k=100$ MIGs for that
$N$. To clarify the rate of convergence and separation of the curves, a
logarithmic scale is used. (The measured errors are adjusted with a finite
population correction of $\sqrt{(1000-N)/(1000-1)}$. The deviation of the
measured errors from prediction near 1000 is believed to be due to limitations
of the finite population correction when applied to a 90\% statistic rather than
a 1-sigma standard devation.)

In order to demonstrate the $1/\sqrt{N}$ behavior of MIG error, Figure
\ref{fig:vanilla_pred_meas}b scales each point from Figure
\ref{fig:vanilla_pred_meas}a by a factor of $\sqrt N$. The flatness of these
curves demonstrates the thesis of this paper, that MIG follows the Law of Large
Numbers, and exhibits error that decreases as $1/\sqrt N$.




\section{Hourglassing\label{hourglassing}}
\subsection{Motivation}
Because commonly-available sensor models (such as RPC) often lack a meaningful
error model to input to MIG, and for computational simplicity, we introduce a
heuristic approach as an alternative to rigorous least-squares MIG. Although
when visualized 'up close' to the true answer, any particular tangle of rays may
seem unclear about where the ground point should be localized, from a distant
perspective, the ray bundle will be something like a cone, widening to the
cluster of satellite (or airborne) perspective centers, narrowing at the ground
point, and widening again beyond the ground point. We seek the answer at the
narrowest point of the cone. In an ideal case, the ray bundle will intersect at
a single point, which has cross-sectional area of 0. In a real-world case, the
bundle will appear as a cone with a `fat' intersection, or an hourglass. Thus
the name `hourglassing' to motivate the heuristic technique.

In `plane-sweep stereo' \cite{PLANE_SWEEP}, a plane is swept through a ground
scene to find stereo focus depth, where conjugate rays are closest. The
technique is expanded for multi-view 3D scene reconstruction in
\cite{PLANE_COLLINS}. We adopt a similar approach here. Given a bundle of rays,
we can intersect the bundle with planes of various heights, compute the
collection of intersections of the ray bundle with each height plane, measure
the spread of the 2-D distribution of points, and choose the plane with the
least spread to be the solution for the height of the desired ground point. For
the horizontal location of the ground point, the natural choice is the mean of
the intersection points in the optimal (spread-minimizing) plane.

Rather than attacking this problem with a brute force search for the
spread-minimizing height by computing intersection sets at very many heights,
and slicing height space sufficiently thin to achieve a desired vertical
resolution, we attempt some theoretical underpinnings for this technique that
will allow a more efficient and precise solution.

\subsection{Computing height of minimum spread}
Assume two heights $z_{+} > z_{-}$, and assume a set of $N$ 3D lines $L_i$, none
of which is horizontal (without loss of generality, we use horizontal planes. If
necessary, the ray bundle can be rotated into a coordinate system so that the
`center' of the ray bundle is vertical). Specify the lines by their
intersections with the planes of heights $z_\pm$ at points
$(x_{+}^{i},y_{+}^{i}, z_+)$ and $(x_{-}^{i},y_{-}^{i}, z_-)$, for $i=1\ldots
N$.

Define
$$x^i(\lambda) = \lambda x^i_+ + (1-\lambda) x^i_-$$
$$y^i(\lambda) = \lambda y^i_+ + (1-\lambda) y^i_-$$
$$z(\lambda) = \lambda z_+ + (1-\lambda) z_-$$ For any $\lambda$,
$(x^i(\lambda), y^i(\lambda), z(\lambda))$ is a point on line $L_i$. Whether the
point is an interpolation between or extrapolation beyond the $z_\pm$ planes
depends on whether $0\le\lambda\le 1$ (in either case, we will just use the term
interpolated). Together, all the points $(x^i(\lambda), y^i(\lambda),
z(\lambda))$ for $i=1\ldots N$ represent the intersection of lines $L_i$ with
the plane with height $z(\lambda)$.

Note that the set of all means $(\bar{x}(\lambda),\bar{y}(\lambda),z(\lambda))$
comprise a line. For the plane at height $z(\lambda)$ that yields the smallest
spread of points $(x^i(\lambda),y^i(\lambda))$ will be the point on that line
which we choose as our answer. (If desired, this mean line between $(\bar{x}(1),
\bar{y}(1),z_+)$ and $(\bar{x}(0),\bar{y}(0),z_-)$ can be computed first, and
the entire bundle rotated so the mean line becomes vertical.)

The two-dimensional spread of the intersection set at a particular $z(\lambda)$
is a symmetric, positive definite, 2x2 covariance matrix:

\[M(\lambda) =
\begin{bmatrix}
 var(x^i(\lambda)) && covar(x^i(\lambda),y^i(\lambda)) \\
  -                && var(y^i(\lambda))
\end{bmatrix}
\]

It can be shown that the covariance of the interpolated points
$(x^i(\lambda), y^i(\lambda))$ can be expressed in terms of variances of and
covariances between the four elementary datasets $x^i_+, x^i_-, y^i_+, y^i_-$ as
follows:

\begin{equation}\label{varx}
var(x^i(\lambda)) = \lambda^2\sigma^2_{x+x+} + \lambda(1-\lambda)(\sigma^2_{x+x-} + \sigma^2_{x-x+}) 
                + (1-\lambda)^2\sigma^2_{x-x-}
\end{equation}
\begin{equation}\label{vary}
var(y^i(\lambda)) = \lambda^2\sigma^2_{y+y+} + \lambda(1-\lambda)(\sigma^2_{y+y-} + \sigma^2_{y-y+}) 
                + (1-\lambda)^2\sigma^2_{y-y-}\end{equation}
\begin{equation}\label{covarxy}
covar(x^i(\lambda),y^i(\lambda)) = \lambda^2\sigma^2_{x+y+}  
                        + \lambda(1-\lambda)(\sigma^2_{x+y-} + \sigma^2_{x-y+}) 
                              + (1-\lambda)^2\sigma^2_{x-y-}
\end{equation}

where
$$\sigma^2_{x+x+}=var(x^i_+),$$
$$\sigma^2_{x+y-}=covar(x^i_+, y^i_-),$$ 
etc. Note in particular that, since all the $\sigma^2$ are functions only of
constants $x^i_\pm, y^i_\pm$, and $N$, each of the 4 elements of $M(\lambda)$ is
a quadratic function of $\lambda$.

If we denote the determinant of a matrix with $|\cdot|$, the area of the
1-sigma error ellipse of $M(\lambda)$ is
$$a(\lambda) = \pi\sqrt{|M(\lambda)|},$$
i.e. the square root of a quartic (4th degree) polynomial.

We seek to minimize $a(\lambda)$, which is equivalent to minimizing
the quartic polynomial 
\begin{eqnarray*}
d(\lambda)&=&|M(\lambda)| \\
          &=&var(x^i(\lambda))var(y^i(\lambda)) - covar(x^i(\lambda),y^i(\lambda))^2
\end{eqnarray*}

Note that, for efficient computation, the quadratic, linear, and
constant coefficients of equations (\ref{varx}-\ref{covarxy}) can be
computed to restate more simply as:
\begin{equation}\label{qx}
var(x^i(\lambda)) = a_x\lambda^2 + b_x\lambda + c_x
\end{equation}
\begin{equation}\label{qy}
var(y^i(\lambda)) = a_y\lambda^2 + b_y\lambda + c_y
\end{equation}
\begin{equation}\label{qxy}
covar(x^i(\lambda),y^i(\lambda)) = a_{xy}\lambda^2 + b_{xy}\lambda + c_{xy}
\end{equation}

Given coefficients computed in equations (\ref{qx}-\ref{qxy}),
$d(\lambda)$ can then be expressed as
\begin{eqnarray}\label{d}
d(\lambda) &=& (a_x\lambda^2 + b_x\lambda + c_x)(a_y\lambda^2 + b_y\lambda + c_y) - 
(a_{xy}\lambda^2 + b_{xy}\lambda + c_{xy})^2 \nonumber \\
&=& (a_xa_y-a_{xy}^2)\lambda^4 + (a_xb_y+b_za_y-2a_{xy}b_{xy})\lambda^3 + \nonumber \\
&& (a_xc_y+b_xb_y+c_xa_y-2a_{xy}c_{xy}-b^2_{xy})\lambda^2 +\\
&& (b_xc_y+c_xb_y-2b_{xy}c_{xy})\lambda + (c_xc_y-c^2_{xy}) \nonumber
\end{eqnarray}

The value $\lambda_{min}$ which minimizes $d(\lambda)$ will also determine the
height $z(\lambda_{min})$ of the output ground point, and then $\lambda_{min}$
can be used to interpolate the intersection set
$(x^i(\lambda_{min}), y^i(\lambda_{min}))$, of which the mean
$(\bar{x}(\lambda_{min}), \bar{y}(\lambda_{min}))$ constitutes the horizontal
component of the output ground point.

\subsection{Empirical Equivalence with MIG}
For every MIG calculation that went into Figures
\ref{fig:vanillaref}--\ref{fig:vanilla_pred_meas}, an Hourglass geolocation
calculation was also performed. Figure \ref{fig:mig_vs_hourglass} shows the
actual errors in the X, Y, and Z directions from a MIG calculation (horizontal
axis) versus an Hourglass calculation (vertical axis), for all of those
calculations. The scatter follows the unity line $y=x$ very closely, showing
that Hourglassing yields very closely the same geolocation as MIG.

\begin{figure}
\includegraphics[width=.8\textwidth]{fig_mig_ply_xyz_sma.png}
\includegraphics[width=.8\textwidth]{fig_mig_ply_xyz_big.png}
\caption{\label{fig:mig_vs_hourglass}MIG vs Hourglass errors. All scales are
  meters. Each row is one graph each for X, Y, and Z errors. Above, 700 points
  are plotted from intersections with $4\le N\le 100$. Below, 26900 points are
  plotted for $10<N<1000$.}
\end{figure}


\subsection{Non-uniqueness}
It is desirable that this quartic polynomial $d(\lambda)$ have no
local minima, but only a single, global minimum. Equivalently, the
cubic derivative should have a single real root and two complex
roots. Unfortunately, there can be degenerate arrangements of image
rays with three real roots of $d'(\lambda$ and multiple minima for
$d(\lambda)$.

%% Consider the following bimodal situation: images of a ground area are captured
%% from satellite positions spaced equally around a horizontal circle, with
%% orientations perfectly intersecting at a ground point at height $z_+$. To this
%% bundle add a duplicate bundle, shifted both horizontally and downwards, to
%% intersect at a lower height $z_-$. Clearly, the spread of the joint bundle at
%% heights $z(\lambda=0)=z_-$ and $z(\lambda=1)=z_+$ are the same. And the spread
%% at $z(\lambda=1/2)$ is somewhat larger (note that the shape of $d(\lambda)$ is
%% depicted sideways to the right). Thus $\lambda={0,1}$ present two minima of
%% $d(\lambda)$, and the hourglassing procedure in this case cannot provide a clear
%% answer.

At least we can compute the exact form of $d(\lambda)$, and with standard
techniques understand clearly whether such a degenerate situation were ever to
present itself. In our empirical testing in section \ref{hsimulation}, we very
seldom encountered such a degenerate case. In the 27600 Hourglassing
computations, 26 degenerate cases were encountered: 16 for $N=4$, 7 for $N=5$,
and 1 each for $N=6,7,8$. In each case, the outer two extrema are local minima,
and the middle is a local maximum, and the ambiguity of the degeneracy was
computed as the difference between the local minimum values of $d(\lambda)$. The
ambiguities ranged from 0.15 to 65. In any case, these are very small numbers of
images, within the scope of this study. Numbers as small as $N=4$ were included
only to provide context and continuity to the results for large numbers of
images. As demonstrated by this study, the chance of degeneracy for a large
number of images is negligible, and in any case it can be detected for manual
review.


\subsection{Error Estimation}
Review of Figures \ref{fig:vanilla_pred_meas}--\ref{fig:vanilla_pred_meas_sqrtn}
suggests an empirical meth\-od of error estimation for Hourglassing. Predicted
error decreases very predictably as $1/\sqrt N$. So an alternative method for
estimating the error of a MIG with $N$ images would be to subsample $M \ll N$
images many times, compute the sample covariance $C_M$ of the $M$-MIGs, and
because of the $1/\sqrt N$ behavior (which is $1/N$ in variance space), and
using a finite population correction, predict the error of the $N$-MIG to
be $$C_N=C_M\frac{M}{N}\cdot\frac{N-1}{N-M}$$

This allows estimation of error apart from the output covariance of the $N$-MIG
(or even of any of the $M$-MIG). There's not much point to this for MIG, since
output covariance is a natural by-product of the MIG algorithm. But for
Hourglassing, this provides a method of error estimation which requires no
apriori error information, simply calculation of additional, smaller Hourglass
calculations on many subamples, and computation and scaling of the sample
covariance of the resulting set of geolocations.

The fundamental principle behind this error estimation technique, is that
apriori error information is not needed for each image in the bundle, because
the bundle {\em is} the distribution of error. For typical collections of only 2
or 4 images, the sample size is too small to reliably know whether the ray
separation is a true representation of the amount of error in the system, or
whether it may be concidentally large (or small). Thus a traditional MIG with
apriori and aposteriori covariance, and reference variance as a consistency
check, is most appropriate. But for large enough collections of images, the
distribution of image rays is a reliable summary of the error of the process.

\subsection{\label{hsimulation}Error Estimation Simulated Results}
To evaluate this empirical method of error estimation for Hourglassing, a subset
of $N=100$ images from $\Pimg$ were chosen, and an Hourglass-geolocation
computed for them. Subsets of $M=N/4$ were sampled from the $N$-set, $k=100$
times, and an Hourglass-geolocation is computed for each $M$-subset. 3x3 sample
covariance was computed for those $k$ ground points, and then scaled by
$1/(N/M)=M/N$ because of the $1/\sqrt{N}$ effect, and corrected also by a FPC
factor of $(N-1)/(N-M)$, to yield the covariance estimate for the
$N$-Hourglass. A MIG was also computed for the same $N$-subset, and its output
covariance computed for reference. This experiment was repeated $R=100$ times.

Figure \ref{fig:mig_vs_hourglass_var} shows the variance of X from MIG vs
Hourglass error estimation. The Hourglass error estimation has more spread than MIG.

\begin{figure}
\centering
\begin{subfigure}{.5\textwidth}
  \centering
  \includegraphics[width=1.2\linewidth]{fig_self_var_xy.png}
\end{subfigure}%
\begin{subfigure}{.5\textwidth}
  \centering
  \includegraphics[width=1.2\linewidth]{fig_self_var_z.png}
\end{subfigure}
\caption{\label{fig:mig_vs_hourglass_var}MIG vs Hourglass estimated
  variances. From left to right, the clusters represent $N=$400, 200 and 100
  images.}
\end{figure}

\section{Conclusion}
In conclusion, we conclude.





\begin{thebibliography}{99} % 2-digit reference nums

\bibitem{LLN} Casella, George, Roger L. Berger, 1990. {\em Statistical
  Inference}, Brooks/Cole, Belmont CA, pp. 214-20.

\bibitem{PLANE_COLLINS}Collins, R.~T., 1996. A space-sweep approach to true
  multi-image matching, In: {\em Proc. CVPR}, 1996.
% https://www.ri.cmu.edu/pub_files/pub1/collins_robert_1996_1/collins_robert_1996_1.pdf

\bibitem{LRO_NINE}Di, K., B.~Xu, B.~Liu, M.~Jia, Z.~Liu, 2016. Geopositioning
  precision analysis of multiple image triangulation using LRO NAC Lunar
  Images, In: {\em Proc. ISPRS}, 2016.
% http://www.int-arch-photogramm-remote-sens-spatial-inf-sci.net/XLI-B4/369/2016/isprs-archives-XLI-B4-369-2016.pdf

\bibitem{BROWSE_DG}Digital Globe Image Finder, {\tt
  http://browse.digitalglobe.com}, accessed 2017-02-24.

\bibitem{MIN}Dolloff, John, and Reuben Settergren, 2010. Worldview-1 Stereo
  Extraction Accuracy With and Without MIN Processing, In: {\em Proc. ASPRS}, 2010.

\bibitem{JEONG_SIX}Jeong, Jaehoon, Chansu Yang, Taejung Kim, 2015. Geo-positioning
  accuracy using multiple-satellite images: IKONOS, QuickBird, and KOMPSAT-2
  Stereo Images, {\em Remote Sensing} 7(4), pp.~4549-64.
% http://www.mdpi.com/2072-4292/7/4/4549

\bibitem{LSQRMIG}Mikhail, Edward M., James S.~Bethel, J.~Chris McGlone,
  2001. {\em Modern Photogrammetry}, Wiley, New York.

\bibitem{SGXP}SOCET GXP, {\tt http://geospatialexploitationproducts.com}

\bibitem{WONNACOT_32_3}Wonnacot, W.~M., 2008. {\em Geolocation with Error Analysis
  Using Imagery from an Experimental Spotlight SAR}, PhD Thesis, Purdue.
% 32 SAR image ``sets'', each a ``three-aperture path sequence''

\bibitem{PLANE_SWEEP}Reference from Joe about ``plane-sweep stereo''

\bibitem{FPC}Finite Population Correction reference, look up something


\end{thebibliography}
 
\end{document}
